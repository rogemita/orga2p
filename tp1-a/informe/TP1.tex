\documentclass[11pt,a4paper,spanish]{article}
\usepackage{a4wide}
\usepackage{listings}
\lstset{language={[x86masm]Assembler}, basicstyle=\small, numbers=left, numberstyle=\tiny}
\usepackage[spanish, activeacute]{babel} 
\usepackage[spanish]{babel}
\usepackage[latin1]{inputenc}
\usepackage{graphicx}
\usepackage{algorithm}
\usepackage{algorithmic}[1]
\frenchspacing
\title{
\begin{centering}
Universidad de Buenos Aires \\
Facultad de Ciencias Exactas y Naturales \\
Departamento de Ciencias de la Computaci\'on \\
\vskip 25pt
\bf Organizaci\'on del computador II \\
\bf Segundo cuatrimestre 2009 
\end{centering}
}
\author{
Grupo: \textsc{POPA} \\
\begin{tabular}[t]{|l|l|l|}
\hline
\textbf{Apellido y nombre} & \textbf{L.U.} & \textbf{mail} \\
\hline 
\hline
Cerrutti, Mariano Javier  & 525/07 & vscorza@gmail.com \\
\hline
Huel, Federico Ariel  & 329/07 & federico.huel@gmail.com \\
\hline
Mita, Rogelio Iv\'an  & 635/07 & rogeliomita@gmail.com \\
\hline
\end{tabular}
}
\date{5 de Octubre de 2009} 
\begin{document}
\maketitle
\tableofcontents

\newpage
\section{Archivos adjuntos}
\textsc{Implementaci'on} 
\begin{itemize}
\item src:
\subitem \textit{bordes.c}
\subitem \textit{asmSobel.asm}
\subitem \textit{asmPrewitt.asm}
\subitem \textit{asmRoberts.asm}
\item img:
\subitem \textit{lena.bmp}
\end{itemize}

\textsc{Includes}
\begin{itemize}
\item \textit{offset.inc} 
\item \textit{macros.mac} 
\end{itemize}

\textsc{Enunciado} 
\begin{itemize}
\item \textit{EnunciadoTP1A.pdf}  
\end{itemize}

\textsc{Informe} 
\begin{itemize}
\item \textit{tp1-a.pdf}  
\end{itemize}

\section{Instrucciones de uso}
Decidimos escribir los c'odigos de las funciones de forma separada para mayor claridad. Utilizamos tambi'en un archivo (\textit{macros.mac}) para definir nuestras macros y  (\textit{offset.inc}) para darle nombres declarativos a los datos de la estructura imagen de opencv. En el cd adjunto al trabajo pr'actico, se encuentran todos los archivos fuente clasificados seg'un el tipo. Decidimos utilizar un makefile para compilar todos los archivos a la vez. 

\newpage
\section{Introducci'on}
Este trabajo consisti'o en escribir en lenguaje ensamblador distintas funciones para buscar bordes de imagen, Tanto en funci'on de x, como en y, o ambas. La raz'on de escribir c'odigo en lenguaje de bajo nivel radica en que es imperativo conocer el manejo interno m'as b'asico de las instrucciones de la arquitectura del computador. \\
\indent  Las funciones en c'odigo assembler estan divididas seg'un el operador que se desea aplicar a la imagen para buscar bordes.

\section{Desarrollo}
Antes de mostrar los c'odigos de cada funci'on explicamos algunas cuestiones generales a todos los algoritmos:
\begin{itemize}
\item Nos fue pedido que en todos los algoritmos sea respetada la convenci'on C, de forma que los mismos pasos (guardado de los registros edi, esi y ebx, ajuste de la pila, etc) se encuentran al principio de cada funci'on.
\item En todas las funciones, consideramos que los par'ametros (excepto los que se ped'ian) no deb'ian ser modificados, y por lo tanto, los mismos los guardamos en variables locales o bien registros. En algunos casos esto es muy importante, por ejemplo, cuando nos pasan un puntero a una lista, si modificamos ese puntero, se pierde la direcci'on de esa lista luego de llamar a la funci'on, y esto no puede ocurrir.
\item Tambi'en tuvimos en cuenta que las im'agenes se guardan en memoria alineadas
\item Por cuestiones de simpleza decidimos en los algoritmos tambi'en procesar la basura de la imagen ya que la libreria opencv cuando hace el resize de la imagen se limita al ancho y alto dado como parametro, y ademas esta decici'on no cambia en absoluto la visualizaci'on de la imagen resultante.
\end{itemize}

A continuaci'on se exponen las funciones implementadas, junto con sus respectivos c'odigos en assembler y se explica a su vez los registros especiales utilizados y su funcionamiento en l'ineas generales.

\subsection{offset.inc}
Aqui estan los offset a datos que usamos de la estructura imagen del openCv, para mayor claridad
\begin{lstlisting}[frame=single]
; typedef struct _IplImage
; {
;     int nSize;
;     int ID;
;     int nChannels;
;     int alphaChannel;
;     int depth;
;     char colorModel[4];
;     char channelSeq[4];
;     int dataOrder;
;     int origin;
;     int align;
;     int width;
;     int height;
;     struct _IplROI *roi;
;     struct _IplImage *maskROI;
;     void *imageId;
;     struct _IplTileInfo *tileInfo;
;     int imageSize;
;     char *imageData;
;     int widthStep;
;     int BorderMode[4];
;     int BorderConst[4];
;     char *imageDataOrigin;
; }
; IplImage;

%define DEPTH		16
%define WIDTH		40
%define HEIGHT		44
%define IMAGE_DATA 	68
%define WIDTH_STEP	72
\end{lstlisting}

\subsection{macros.mac}
Estos son las macros utilizadas en cada funcion.
\begin{lstlisting}[frame=single]
;========================
; MACRO doEnter
;========================
; Escribe el encabezado que arma el stack frame
; Entrada:
;	tamanio		tamanio de memoria a reservar al entrar al proc
;========================
%macro doEnter 0-1 0
	push ebp
	mov ebp, esp
%if %1 <> 0
	sub esp, %1
%endif
	push edi
	push esi
	push ebx
%endmacro
;========================
; MACRO doLeave
;========================
; Escribe la salida que restaura el stack frame previo
; Entrada:
;	tamanio		tamanio de memoria reservada al entrar al proc
;	doRet		se marca en 1 si debe llamar a ret
;========================
%macro doLeave 0-2 0,0
	pop ebx
	pop esi
	pop edi
%if %1 <> 0
	add esp, %1
%endif
	pop ebp
%if %2 <> 0
	ret
%endif	
%endmacro

;========================
; MACRO doWrite
;========================
; Escribe una cadena a consola
; Entrada: 
;	mensaje		direccion de comienzo de la cadena
;	len		largo de la cadena a escribir
;========================
%macro doWrite 1
	%%msg: db %1
	%%len: equ $- %%msg	
	mov eax,4			;inicializa escritura a consola
	mov ebx,1
	mov ecx,%%msg
	mov edx,%%len
	int 80h
%endmacro
;========================
; MACRO doEnd
;========================
; Termina la ejecucion con el codigo deseado
; Entrada: 
;	codigo		codigo de error deseado, cero en su defecto
;========================
%macro doEnd 0-1 0
	mov eax,1
	mov ebx,%1
	int 80h
%endmacro
;========================
; MACRO doMalloc
;========================
; Pide la cantidad especificada de memoria
; Entrada:
;	cantidad	cantidad de memoria a reserver
;========================
%macro doMalloc 1
	push %1
	call malloc
	add esp, 4
%endmacro
;========================
; MACRO doRetc
;========================
; Retorna si se cumple la condicion especificada
; Entrada:
;	condicion	condicion ante la cual retornar
;========================
%macro  doRetc 1 
        j%-1    %%skip 
        ret 
  %%skip: 
%endmacro
;========================
; MACRO doWriteFile
;========================
; Escribe a archivo solo si esta definido el DEBUG
; Entrada:
;	arch		puntero al archivo
;	msg		texto a escribir
; Uso:
;	doWriteFile [FHND], {"hola mundo",13,10}
;========================
%macro  doWriteFile 2+ 
%ifdef DEBUG
        jmp     %%endstr 
  %%str:        db      %2 
  %%endstr: 
        mov     dx,%%str 
        mov     cx,%%endstr-%%str 
        mov     bx,%1 
        mov     ah,0x40 
        int     0x21 
%endif
%endmacro

\end{lstlisting}

\newpage
\subsection{Recorrido de la matriz de imagen}
\begin{enumerate}
\item \textbf{Cosas que tuvimos en cuenta} 
\subitem La matriz la recorrimos procesando la basura \\
\subitem Cada pixel ocupa exactamente un byte, ya que la imagen que recive la funci'on es en escala de grises  \\
\subitem En la arquitectura IA-32 los datos en memoria son guardados de forma \emph{little-endian}, de forma que el byte m'as significativo de un dato se encuetra en la posici'on de memoria m'as alta.
\end{enumerate}

\subsection{Funciones implementadas}
\begin{itemize}
\item Intentamos reducir el acceso a memoria utilizando registros para utilizar en aquellas operaciones que se encontraban dentro de los ciclos, para evitar multiples accesos a memoria. Si bien esto llev'o a tener un codigo menos entendible a simple vista, lo explicaremos en detalle.
\item Variables locales
\subitem \textit{SRC} Contiene el puntero a la imagen de entrada.
\subitem \textit{DST} Contiene el puntero a la imagen resultado.
\subitem \textit{WIDTH} Contiene el tamano de ancho de la imagen.
\subitem \textit{HEIGHT} Contiene la altura de la imagen.
\subitem \textit{XORDER} Variable que especifica si el calculo debe hacerse sobre la derivada X.
\subitem \textit{YORDER} Variable que especifica si el calculo debe hacerse sobre la derivada Y.
\subitem \textit{T\_HEIGHT} Variable que se usa como contador para el recorrido vertical.
\item Registros de prop'osito general 
\subitem \textit{eax} Se utiliza como acumulador, para realizar los calculos de los pixeles.
\subitem \textit{ebx} Se utiliza como acumulador, para realizar los calculos de los pixeles.
\subitem \textit{ecx} Se utiliza para recorrer la imagen de entrada.
\subitem \textit{edx} Se utiliza como contador para el recorrido horizontal de la imagen.
\subitem \textit{esi} Se utiliza para recorrer la imagen de salida.
\subitem \textit{edi} Contiene el ancho de la imagen.
\item \textbf{Funcionamiento}
\subitem Lo primero que hacen todos los algoritmos es ubicar al registro ecx en el comienzo de la imagen de entrada y a esi en el primer pixel que se modificar'a en la imagen de salida.
Luego comienza el ciclo que recorre la matriz verticalmente, para lo que carga en edx el ancho de la imagen,para usarlo como contador del recorrido horizontal, sac'andole los pixeles laterales ya que no se procesar'an debido al c'alculo.
\end{itemize}

\subsubsection{Sobel} 
\begin{enumerate}
\item \textbf{Cosas que tuvimos en cuenta}
\subitem Lo que tuvimos en cuenta si tuvimos algo.
\item \textbf{X Sobel}
\subitem Dentro del ciclo que recorre la imagen horizontalmente, se copian en eax los cuatro pixeles que se encuentran a continuaci�n en la imagen y mediante una m�scara se obtienen s�lo el primero y el tercero. Luego se hace lo mismo para los pixeles que se encuentran en la siguiente l�nea guard�ndolos en ebx, y shifte�ndolos un bit a izquierda para duplicar su valor. Se suman ambos registros y y se gurada el resultado en eax.
Se cargan los pixeles de la tercer l�nea en ebx, se le aplica una m�scara para obtener s�lo los necesarios y se los suma a los acumulados quedando en eax la parte alta de eax la suman de los primeros pixeles
de cada fila y en la parte alta los terceros. Luego, se copia  eax a ebx, y se lo shiftea (a eax)para que queden los pixeles que deben restarse en la parte menos significativa. 
Con los datos listos, se realiza la resta.
\item \textbf{Y Sobel}
\subitem Dentro del ciclo que recorre la imagen horizontalmente, se cargan en eax y en ebx 4 pixeles en los que los primeros 3 contienen en orden los pixeles que ser�n sumados de acuerdo al c'alculo de Sobel.
Luego se aplica una m'ascara a eax que deja solo el primer y tercer pixel y otra en ebx que deja solo el segundo. Se hace un shift a derecha a ebx para ubicar el pixel en la parte baja del registro para realizar la suma
con los otros pixeles, pero dejando un bit a derecha para duplicar su valor. A continuaci'on se suman el primer y segundo(duplicado en valor) pixel, se shiftea el tercero y se lo suma a los dos anteriores, quedando en el
registro ebx el valor de la suma de los pixeles.
Luego se copian en eax los cuatro pixeles que contienen a los que deben restarse y se elimina el que no se necesita mediante la utilizaci�n de una m'ascara. Para poder operar, se utiliza la funci'on ror para dejar en la word menos significativa solo un pixel y poder restarlo llamando a ax. Luego se borra la parte baja de eax y se guarda la parte alta en ebx (La parte alta de ebx no estaba siendo utilizada ya que acumula solo la suma de tres Words). Se shiftea eax para que quede el tercer pixel en la parte baja y se lo resta al acumulador. Lo mismo se hace con el segundo pixel, pero dejando un bit a derecha al shiftearlo para que su valor sea el doble.
\item \textbf{Ambas derivadas}
\subitem Antes de copiarlo, se comprueba que el pixel calculado no se haya saturado, y en caso de que eso haya sucedido lo que se copia es el mayor o menor valor para el pixel, de acuerdo a la saturacion, evitando desbordes.
Luego se restan los contadores corresopondientes y se comprueba si ha finalizado alguno de los ciclos para decidir que salto se raliza.
\item \textbf{C'odigo}
\end{enumerate}
\begin{lstlisting}[frame=single]
ACA VA EL CODIGO
\end{lstlisting}

\subsubsection{Prewitt} 
\begin{enumerate}
\item \textbf{Cosas que tuvimos en cuenta}
\subitem Lo que tuvimos en cuenta si tuvimos algo.
\item \textbf{X Prewitt}
\subitem Dentro del ciclo horizontal, se copian en eax los cuatro pixeles que se encuentran a continuaci'on en la imagen y mediante una m'ascara se obtienen solo el primero y el tercero. Luego se hace
lo mismo para los pixeles que se encuentran en la siguiente linea guard'andolos en ebx. Se suman ambos registros y y se gurada el resultado en eax.
Se cargan los pixeles de la tercer l�nea en ebx, se le aplica una m'ascara para obtener solo los necesarios y se los suma a los acumulados quedando en eax la parte alta de eax la suman de los primeros pixeles
de cada fila y en la parte alta los terceros. Luego, se copia  eax a ebx, y se lo shiftea 16 bits a derecha(a eax)para que queden los pixeles que deben restarse en la parte menos significativa. 
Con los datos listos, se realiza la resta.
\item \textbf{Y Prewitt}
\subitem Dentro del ciclo que recorre la imagen horizontalmente, se cargan en eax y en ebx 4 pixeles en los que los primeros 3 contienen en orden los pixeles que ser'an sumados de acuerdo al c'alculo de Sobel.
Luego se aplica una m'ascara a eax que deja solo el primer y tercer pixel y otra en ebx que deja solo el segundo. Se hace un shift a derecha a ebx para ubicar el pixel en la parte baja del registro para realizar la suma con los otros pixeles. A continuaci'on se suman el primer y segundo pixel, se shiftea el tercero y se lo suma a los dos anteriores, quedando en el registro ebx el valor de la suma de los pixeles.
Luego se copian en eax los cuatro pixeles que contienen a los que deben restarse y se elimina el que no se necesita mediante la utilizaci'on de una m'ascara. Para poder operar, se utiliza la funci'on ror para dejar en la word menos significativa solo un pixel y poder restarlo llamando a ax. Luego se borra la parte baja de eax y se guarda la parte alta en ebx (La parte alta de ebx no estaba siendo utilizada ya que acumula solo la suma de tres Words). Se shiftea eax para que quede el tercer pixel en la parte baja y se lo resta al acumulador. Lo mismo se hace con el segundo pixel, pero dejando un bit a derecha al shiftearlo para que su valor sea el doble.
\item \textbf{Ambas derivadas}
\subitem Antes de copiarlo, se comprueba que el pixel calculado no se haya saturado, y en caso de que eso haya sucedido lo que se copia es el mayor o menor valor para el pixel, de acuerdo a la saturacion, evitando desbordes.
Luego se restan los contadores corresopondientes y se comprueba si ha finalizado alguno de los ciclos para decidir que salto se raliza.
\item \textbf{C'odigo}
\end{enumerate}
\begin{lstlisting}[frame=single]
ACA VA EL CODIGO
\end{lstlisting}

\subsubsection{Roberts} 
\begin{enumerate}
\item \textbf{Cosas que tuvimos en cuenta}
\subitem Lo que tuvimos en cuenta si tuvimos algo.
\item \textbf{X Roberts}
\subitem Comienza el ciclo horizontal. Se cargan los tres pixeles a los que est'a apuntando ecx (que es el que recorre la imagen) y se utiliza una m'ascara para quedarse solo con el primer pixel en ax, ya que es el que se utilizar'a en el c'alculo.
Se copian en bx los dos pixeles que se encuentran en la siguiente linea, se lo enmascara para quedarse con el m'as significativo y se lo shiftea para poder operar con el pixel guardado en ax.
Luego se realiza la resta entre bx y ax, se verifica si hay saturaci'on y se copia el pixel guardado en al en la imagen de salida.
\item \textbf{Y Roberts}
\subitem En el ciclo horizontal, se copian en bx los pixeles necesarios qued'andose solo con el segundo, que es el que se utilizar�. Se lo shiftea 8 bits a derecha para que quede en la parte menos significativa y que se pueda operar con otro pixel. Luego se copia el pixel que est'a en la linea siguiente en ax usando una m'ascara para filtrar el byte m'as significativo y se realiza la resta entre bx y ax. Se verifica la aparici'on de saturaci'on y se copia el pixel en lugar correspondiente de la imagen de salida.
Se incrementa los registros que recorren las im�genes y se aumenta de l�nea si no deben escribirse m'as pixeles en dicha l�nea y vuelve a comenzar el ciclo. 
\item \textbf{Ambas derivadas}
\subitem Antes de copiarlo, se comprueba que el pixel calculado no se haya saturado, y en caso de que eso haya sucedido lo que se copia es el mayor o menor valor para el pixel, de acuerdo a la saturacion, evitando desbordes.
Luego se restan los contadores corresopondientes y se comprueba si ha finalizado alguno de los ciclos para decidir que salto se raliza.
\item \textbf{C'odigo}
\end{enumerate}
\begin{lstlisting}[frame=single]
ACA VA EL CODIGO
\end{lstlisting}

\newpage
\section{Resultados}
COMPLETAR CON RESULTADOS

\section{Conclusiones}
Luego de implementar este trabajo en assembler, podemos concluir que si bien nos da una gran libertad a la hora de programar (modificar directamente la memoria, trabajar con los registros directamente, etc.), la claridad del c'odigo de estos programas no es tan claro como los de lenguaje de alto nivel que manejamos \textit{C}, \textit{C++}, etc. Vemos esto como una desventaja ya que teniendo en cuenta que el trabajo es grupal, en ocasiones tuvimos que repartir las tareas, es decir, algunas funciones, y luego, al juntarnos y ver las implementaciones que hab'ia hecho cada uno se dificultaba entenderlas. \\
De todas formas, consideramos que la realizaci'on de este trabajo nos sirvi'o para obtener un importante conocimiento sobre el lenguaje de bajo nivel, el cual complementa entender cosas de lenguajes de m'as alto nivel, y consideramos adem'as que es una muy buena pr'actica para ayudarnos a comprender el funcionamiento interno de un computador.

\end{document}